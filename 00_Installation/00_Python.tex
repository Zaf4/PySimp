% Options for packages loaded elsewhere
\PassOptionsToPackage{unicode}{hyperref}
\PassOptionsToPackage{hyphens}{url}
\PassOptionsToPackage{dvipsnames,svgnames,x11names}{xcolor}
%
\documentclass[
  letterpaper,
  DIV=11,
  numbers=noendperiod]{scrreprt}

\usepackage{amsmath,amssymb}
\usepackage{iftex}
\ifPDFTeX
  \usepackage[T1]{fontenc}
  \usepackage[utf8]{inputenc}
  \usepackage{textcomp} % provide euro and other symbols
\else % if luatex or xetex
  \usepackage{unicode-math}
  \defaultfontfeatures{Scale=MatchLowercase}
  \defaultfontfeatures[\rmfamily]{Ligatures=TeX,Scale=1}
\fi
\usepackage{lmodern}
\ifPDFTeX\else  
    % xetex/luatex font selection
\fi
% Use upquote if available, for straight quotes in verbatim environments
\IfFileExists{upquote.sty}{\usepackage{upquote}}{}
\IfFileExists{microtype.sty}{% use microtype if available
  \usepackage[]{microtype}
  \UseMicrotypeSet[protrusion]{basicmath} % disable protrusion for tt fonts
}{}
\makeatletter
\@ifundefined{KOMAClassName}{% if non-KOMA class
  \IfFileExists{parskip.sty}{%
    \usepackage{parskip}
  }{% else
    \setlength{\parindent}{0pt}
    \setlength{\parskip}{6pt plus 2pt minus 1pt}}
}{% if KOMA class
  \KOMAoptions{parskip=half}}
\makeatother
\usepackage{xcolor}
\setlength{\emergencystretch}{3em} % prevent overfull lines
\setcounter{secnumdepth}{5}
% Make \paragraph and \subparagraph free-standing
\makeatletter
\ifx\paragraph\undefined\else
  \let\oldparagraph\paragraph
  \renewcommand{\paragraph}{
    \@ifstar
      \xxxParagraphStar
      \xxxParagraphNoStar
  }
  \newcommand{\xxxParagraphStar}[1]{\oldparagraph*{#1}\mbox{}}
  \newcommand{\xxxParagraphNoStar}[1]{\oldparagraph{#1}\mbox{}}
\fi
\ifx\subparagraph\undefined\else
  \let\oldsubparagraph\subparagraph
  \renewcommand{\subparagraph}{
    \@ifstar
      \xxxSubParagraphStar
      \xxxSubParagraphNoStar
  }
  \newcommand{\xxxSubParagraphStar}[1]{\oldsubparagraph*{#1}\mbox{}}
  \newcommand{\xxxSubParagraphNoStar}[1]{\oldsubparagraph{#1}\mbox{}}
\fi
\makeatother

\usepackage{color}
\usepackage{fancyvrb}
\newcommand{\VerbBar}{|}
\newcommand{\VERB}{\Verb[commandchars=\\\{\}]}
\DefineVerbatimEnvironment{Highlighting}{Verbatim}{commandchars=\\\{\}}
% Add ',fontsize=\small' for more characters per line
\usepackage{framed}
\definecolor{shadecolor}{RGB}{241,243,245}
\newenvironment{Shaded}{\begin{snugshade}}{\end{snugshade}}
\newcommand{\AlertTok}[1]{\textcolor[rgb]{0.68,0.00,0.00}{#1}}
\newcommand{\AnnotationTok}[1]{\textcolor[rgb]{0.37,0.37,0.37}{#1}}
\newcommand{\AttributeTok}[1]{\textcolor[rgb]{0.40,0.45,0.13}{#1}}
\newcommand{\BaseNTok}[1]{\textcolor[rgb]{0.68,0.00,0.00}{#1}}
\newcommand{\BuiltInTok}[1]{\textcolor[rgb]{0.00,0.23,0.31}{#1}}
\newcommand{\CharTok}[1]{\textcolor[rgb]{0.13,0.47,0.30}{#1}}
\newcommand{\CommentTok}[1]{\textcolor[rgb]{0.37,0.37,0.37}{#1}}
\newcommand{\CommentVarTok}[1]{\textcolor[rgb]{0.37,0.37,0.37}{\textit{#1}}}
\newcommand{\ConstantTok}[1]{\textcolor[rgb]{0.56,0.35,0.01}{#1}}
\newcommand{\ControlFlowTok}[1]{\textcolor[rgb]{0.00,0.23,0.31}{\textbf{#1}}}
\newcommand{\DataTypeTok}[1]{\textcolor[rgb]{0.68,0.00,0.00}{#1}}
\newcommand{\DecValTok}[1]{\textcolor[rgb]{0.68,0.00,0.00}{#1}}
\newcommand{\DocumentationTok}[1]{\textcolor[rgb]{0.37,0.37,0.37}{\textit{#1}}}
\newcommand{\ErrorTok}[1]{\textcolor[rgb]{0.68,0.00,0.00}{#1}}
\newcommand{\ExtensionTok}[1]{\textcolor[rgb]{0.00,0.23,0.31}{#1}}
\newcommand{\FloatTok}[1]{\textcolor[rgb]{0.68,0.00,0.00}{#1}}
\newcommand{\FunctionTok}[1]{\textcolor[rgb]{0.28,0.35,0.67}{#1}}
\newcommand{\ImportTok}[1]{\textcolor[rgb]{0.00,0.46,0.62}{#1}}
\newcommand{\InformationTok}[1]{\textcolor[rgb]{0.37,0.37,0.37}{#1}}
\newcommand{\KeywordTok}[1]{\textcolor[rgb]{0.00,0.23,0.31}{\textbf{#1}}}
\newcommand{\NormalTok}[1]{\textcolor[rgb]{0.00,0.23,0.31}{#1}}
\newcommand{\OperatorTok}[1]{\textcolor[rgb]{0.37,0.37,0.37}{#1}}
\newcommand{\OtherTok}[1]{\textcolor[rgb]{0.00,0.23,0.31}{#1}}
\newcommand{\PreprocessorTok}[1]{\textcolor[rgb]{0.68,0.00,0.00}{#1}}
\newcommand{\RegionMarkerTok}[1]{\textcolor[rgb]{0.00,0.23,0.31}{#1}}
\newcommand{\SpecialCharTok}[1]{\textcolor[rgb]{0.37,0.37,0.37}{#1}}
\newcommand{\SpecialStringTok}[1]{\textcolor[rgb]{0.13,0.47,0.30}{#1}}
\newcommand{\StringTok}[1]{\textcolor[rgb]{0.13,0.47,0.30}{#1}}
\newcommand{\VariableTok}[1]{\textcolor[rgb]{0.07,0.07,0.07}{#1}}
\newcommand{\VerbatimStringTok}[1]{\textcolor[rgb]{0.13,0.47,0.30}{#1}}
\newcommand{\WarningTok}[1]{\textcolor[rgb]{0.37,0.37,0.37}{\textit{#1}}}

\providecommand{\tightlist}{%
  \setlength{\itemsep}{0pt}\setlength{\parskip}{0pt}}\usepackage{longtable,booktabs,array}
\usepackage{calc} % for calculating minipage widths
% Correct order of tables after \paragraph or \subparagraph
\usepackage{etoolbox}
\makeatletter
\patchcmd\longtable{\par}{\if@noskipsec\mbox{}\fi\par}{}{}
\makeatother
% Allow footnotes in longtable head/foot
\IfFileExists{footnotehyper.sty}{\usepackage{footnotehyper}}{\usepackage{footnote}}
\makesavenoteenv{longtable}
\usepackage{graphicx}
\makeatletter
\def\maxwidth{\ifdim\Gin@nat@width>\linewidth\linewidth\else\Gin@nat@width\fi}
\def\maxheight{\ifdim\Gin@nat@height>\textheight\textheight\else\Gin@nat@height\fi}
\makeatother
% Scale images if necessary, so that they will not overflow the page
% margins by default, and it is still possible to overwrite the defaults
% using explicit options in \includegraphics[width, height, ...]{}
\setkeys{Gin}{width=\maxwidth,height=\maxheight,keepaspectratio}
% Set default figure placement to htbp
\makeatletter
\def\fps@figure{htbp}
\makeatother

\KOMAoption{captions}{tableheading}
\makeatletter
\@ifpackageloaded{bookmark}{}{\usepackage{bookmark}}
\makeatother
\makeatletter
\@ifpackageloaded{caption}{}{\usepackage{caption}}
\AtBeginDocument{%
\ifdefined\contentsname
  \renewcommand*\contentsname{Table of contents}
\else
  \newcommand\contentsname{Table of contents}
\fi
\ifdefined\listfigurename
  \renewcommand*\listfigurename{List of Figures}
\else
  \newcommand\listfigurename{List of Figures}
\fi
\ifdefined\listtablename
  \renewcommand*\listtablename{List of Tables}
\else
  \newcommand\listtablename{List of Tables}
\fi
\ifdefined\figurename
  \renewcommand*\figurename{Figure}
\else
  \newcommand\figurename{Figure}
\fi
\ifdefined\tablename
  \renewcommand*\tablename{Table}
\else
  \newcommand\tablename{Table}
\fi
}
\@ifpackageloaded{float}{}{\usepackage{float}}
\floatstyle{ruled}
\@ifundefined{c@chapter}{\newfloat{codelisting}{h}{lop}}{\newfloat{codelisting}{h}{lop}[chapter]}
\floatname{codelisting}{Listing}
\newcommand*\listoflistings{\listof{codelisting}{List of Listings}}
\makeatother
\makeatletter
\makeatother
\makeatletter
\@ifpackageloaded{caption}{}{\usepackage{caption}}
\@ifpackageloaded{subcaption}{}{\usepackage{subcaption}}
\makeatother

\ifLuaTeX
  \usepackage{selnolig}  % disable illegal ligatures
\fi
\usepackage{bookmark}

\IfFileExists{xurl.sty}{\usepackage{xurl}}{} % add URL line breaks if available
\urlstyle{same} % disable monospaced font for URLs
\hypersetup{
  pdftitle={PySimp},
  pdfauthor={Zafer Kosar},
  colorlinks=true,
  linkcolor={blue},
  filecolor={Maroon},
  citecolor={Blue},
  urlcolor={Blue},
  pdfcreator={LaTeX via pandoc}}


\title{PySimp}
\author{Zafer Kosar}
\date{2024-06-09}

\begin{document}
\maketitle

\renewcommand*\contentsname{Table of contents}
{
\hypersetup{linkcolor=}
\setcounter{tocdepth}{2}
\tableofcontents
}

\bookmarksetup{startatroot}

\chapter*{Preface}\label{preface}
\addcontentsline{toc}{chapter}{Preface}

\markboth{Preface}{Preface}

This is a Quarto book.

To learn more about Quarto books visit
\url{https://quarto.org/docs/books}.

\part{Installation}

\chapter{Installing python}\label{installing-python}

It is suggested use panultimate (one before the latest) major release of
Python. Python 3 is now the golden standard and major releases follow
3.x.x convetion. So, if the latest release is \textbf{3.12.x}, use the
\textbf{3.11.y}, y being the latest version, the reason being
compatibility with 3rd party libraries takes some time. For this book,
we will use Python 3.11.9.

\section{Windows}\label{windows}

On windows just go to https://www.python.org/downloads/ and scroll down
to ``Python releases by version number:'' section and select a version
to download. And continue with the installation.

On the installation, click ``\textbf{Add python.exe to PATH}'' so you
can use Python on your terminal.

\section{Mac}\label{mac}

On Mac just go to https://www.python.org/downloads/ and scroll down to
``Python releases by version number:'' section and select a version to
download. And continue with the installation.

\chapter{Installing Third Party
Libraries}\label{installing-third-party-libraries}

Effective use of Python requires many third party libraries (packages).
Most prominently \texttt{numpy} is used for working with arrays,
\texttt{pandas} or \texttt{polars} for tabular data, \texttt{scipy}
contains scientific algorithms etc., These third party libraries make
Python very attractive as they solve the \textbf{speed problem} of
Python while utilizing the ease-of-use of Python.

\section{pip}\label{pip}

To install third party libraries, we can use \texttt{pip} installer that
comes built-in with Python installation. We add \texttt{install}
\textbf{flag} for installation to declare our purpose. Lastly,
\texttt{install} flag requires a parameter which is the library name
e.g., \texttt{numpy}. Shortly, we can open a \textbf{terminal window}
and write the following.

\begin{Shaded}
\begin{Highlighting}[]
\NormalTok{pip install numpy}
\end{Highlighting}
\end{Shaded}

This will directly install \texttt{numpy} from PyPi, the public
repository of Python, to your Python packages.

! Note That in some cases \texttt{pip3} is used instead of \texttt{pip}

\subsection{Installing Libraries on
Notebook}\label{installing-libraries-on-notebook}

In some cases you may want to or have to install libraries inside of a
Jupyter Notebook or Notebook in general. In such cases, just add
\textbf{exclammation mark} ``!'' or \textbf{percentage sign} ``\%''
before \texttt{pip}.

\begin{Shaded}
\begin{Highlighting}[]
\OperatorTok{\%}\NormalTok{pip install numpy}
\end{Highlighting}
\end{Shaded}

\subsection{Installing without admin
priviliges}\label{installing-without-admin-priviliges}

If you're working on an HPC or something similar you may not have admin
priviliges. Therefore, you may not have to access to main folder of
Python packages. As a work-around you can add \texttt{-\/-user} flag to
install the packages to your own packages.

\begin{Shaded}
\begin{Highlighting}[]
\NormalTok{pip install numpy {-}{-}user}
\end{Highlighting}
\end{Shaded}

Above Python 3.10 installation defaults to user, so you don't need the
\texttt{-\/-user} flag.

\part{Basics}

\chapter{Variables}\label{variables}

A variable is used as shortcut for a \textbf{value}. Python has some
fundamental variable types to hold distinct type of values.

The most notable of these fundamental variable types are;

\section{Strings}\label{strings}

\textbf{Strings} ,that are denoted by type \texttt{str} in python, used
to hold \emph{string} of characters.

\begin{Shaded}
\begin{Highlighting}[]
\NormalTok{my\_name }\OperatorTok{=} \StringTok{"Zafer Kosar the First of his name"}
\end{Highlighting}
\end{Shaded}

\section{Floats and Integers}\label{floats-and-integers}

\textbf{Numeric} variables \texttt{float} for floating point numbers
(decimal numbers) and \texttt{int} for integers (whole numbers).

\begin{Shaded}
\begin{Highlighting}[]
\NormalTok{number\_of\_my\_friends }\OperatorTok{=} \DecValTok{12} \CommentTok{\# I can\textquotesingle{}t have a 0.35 friend.}
\BuiltInTok{type}\NormalTok{(number\_of\_my\_friends) }\CommentTok{\# this will output the type of the variable}
\end{Highlighting}
\end{Shaded}

\begin{verbatim}
int
\end{verbatim}

\begin{Shaded}
\begin{Highlighting}[]
\NormalTok{my\_height\_in\_meters }\OperatorTok{=} \FloatTok{1.84} \CommentTok{\# Some variables require floating point numbers}
\BuiltInTok{type}\NormalTok{(my\_height\_in\_meters)}
\end{Highlighting}
\end{Shaded}

\begin{verbatim}
float
\end{verbatim}

\section{Boolean}\label{boolean}

\textbf{Logic} based variables holds \texttt{True} or \texttt{False}
\textbf{boolean} values denoted by \texttt{bool}. They can be used to
assess equality of values and variables. They are very useful to check
whether the \textbf{conditions} are satisfied, thus control the flow of
the code.

As single equal sign \texttt{=} is reserved for assignment of value to
variable. Double equal \texttt{==} sign is used for comparison.

An example comparison that will output \texttt{False};

\begin{Shaded}
\begin{Highlighting}[]
\DecValTok{1} \OperatorTok{==} \DecValTok{0}
\end{Highlighting}
\end{Shaded}

\begin{verbatim}
False
\end{verbatim}

Also, variables can be used for comparison;

\begin{Shaded}
\begin{Highlighting}[]
\NormalTok{my\_height\_in\_meters }\OperatorTok{==}\NormalTok{ my\_name}
\end{Highlighting}
\end{Shaded}

\begin{verbatim}
False
\end{verbatim}

As for the \texttt{type} of this assessment;

\begin{Shaded}
\begin{Highlighting}[]
\BuiltInTok{type}\NormalTok{(my\_height\_in\_meters }\OperatorTok{==}\NormalTok{ my\_name)}
\end{Highlighting}
\end{Shaded}

\begin{verbatim}
bool
\end{verbatim}

An example that will output \texttt{True};

\begin{Shaded}
\begin{Highlighting}[]
\DecValTok{2} \OperatorTok{==} \FloatTok{2.0} \CommentTok{\# ! Important Note here: Unlike some other languages, in Python float 2 (2.0) is equal to int 2 (2).}
\end{Highlighting}
\end{Shaded}

\begin{verbatim}
True
\end{verbatim}

As expected

\begin{Shaded}
\begin{Highlighting}[]
\BuiltInTok{type}\NormalTok{(}\DecValTok{2}\NormalTok{) }\OperatorTok{==} \BuiltInTok{type}\NormalTok{(}\FloatTok{2.0}\NormalTok{) }\CommentTok{\# int vs float}
\end{Highlighting}
\end{Shaded}

\begin{verbatim}
False
\end{verbatim}

\section{None}\label{none}

\texttt{None},as the name suggests, is used for denoting that there is
no value assigned to the variable. Note that it is different from not
being defined at all.

We haven't talked about the functions yet. But, let's declare a simple
function named \texttt{simple\_function()} which returns
\textbf{nothing}.

\begin{Shaded}
\begin{Highlighting}[]
\KeywordTok{def}\NormalTok{ simple\_function():}
    \ControlFlowTok{return}
\end{Highlighting}
\end{Shaded}

\begin{Shaded}
\begin{Highlighting}[]
\NormalTok{x }\OperatorTok{=}\NormalTok{ simple\_function()}
\BuiltInTok{type}\NormalTok{(x)}
\end{Highlighting}
\end{Shaded}

\begin{verbatim}
NoneType
\end{verbatim}

We can also declare a variable with a value of \texttt{None} to make it
\texttt{NoneType}.

\begin{Shaded}
\begin{Highlighting}[]
\NormalTok{y }\OperatorTok{=} \VariableTok{None}
\BuiltInTok{type}\NormalTok{(y)}
\end{Highlighting}
\end{Shaded}

\begin{verbatim}
NoneType
\end{verbatim}

\section{About variables}\label{about-variables}

Variable names \textbf{must not} start with a number and \textbf{must
not} include special characters except for underscore ``\_''. Python
uses \emph{snake case} convention which states variable names (and each
word in a variable name) \textbf{should} start with a lowercase letter
and each word should be seperated by an underscore `\_'.

Python is a \textbf{dynamically typed} language. So, variables types
will be inferred by the interpreter. Although this makes Python a slower
compared to \textbf{statically typed} languages (e.g., C, Rust, Java
etc.,), it makes Python beginner friendly also reduces the development
time.

\begin{Shaded}
\begin{Highlighting}[]
\NormalTok{num\_samples : }\BuiltInTok{int} \OperatorTok{=} \DecValTok{33} \CommentTok{\# static typing}
\NormalTok{num\_samples }\OperatorTok{=} \StringTok{"33"} \CommentTok{\# dynamic typing}
\end{Highlighting}
\end{Shaded}

Although Python allows static typing (aka type hinting), it is
\textbf{not required} and interpreter \textbf{ignores} it. However, they
may be useful for the programmer to keep track of their variables.

\chapter{Lists}\label{lists}

Lists are used to store mutable \textbf{sequence} or \textbf{collection}
of values (or items).

A list can be created with collection of values seperated by commas
\texttt{,} and enclosed within \texttt{{[}{]}}

\begin{Shaded}
\begin{Highlighting}[]
\NormalTok{ids }\OperatorTok{=}\NormalTok{ [}\DecValTok{0}\NormalTok{, }\DecValTok{1}\NormalTok{, }\DecValTok{2}\NormalTok{, }\DecValTok{3}\NormalTok{, }\DecValTok{4}\NormalTok{, }\DecValTok{5}\NormalTok{, }\DecValTok{6}\NormalTok{, }\DecValTok{7}\NormalTok{, }\DecValTok{8}\NormalTok{, }\DecValTok{9}\NormalTok{, }\DecValTok{10}\NormalTok{]}
\NormalTok{ids}
\end{Highlighting}
\end{Shaded}

\begin{verbatim}
[0, 1, 2, 3, 4, 5, 6, 7, 8, 9, 10]
\end{verbatim}

\section{List methods}\label{list-methods}

Items also can be appended to the \textbf{end} of the list with
\texttt{append()} method of list object.

\begin{Shaded}
\begin{Highlighting}[]
\NormalTok{ids.append(}\DecValTok{11}\NormalTok{)}
\NormalTok{ids}
\end{Highlighting}
\end{Shaded}

\begin{verbatim}
[0, 1, 2, 3, 4, 5, 6, 7, 8, 9, 10, 11]
\end{verbatim}

or they can also be appended to the beginning or any given (existing)
index of a list via \texttt{insert()} method.

\begin{Shaded}
\begin{Highlighting}[]
\NormalTok{ids.insert(}\DecValTok{0}\NormalTok{, }\OperatorTok{{-}}\DecValTok{1}\NormalTok{) }\CommentTok{\# at index 0, insert {-}1}
\NormalTok{ids}
\end{Highlighting}
\end{Shaded}

\begin{verbatim}
[-1, 0, 1, 2, 3, 4, 5, 6, 7, 8, 9, 10, 11]
\end{verbatim}

It is possible to reverse a list via \texttt{reverse()} method.

\begin{Shaded}
\begin{Highlighting}[]
\NormalTok{ids.reverse()}
\NormalTok{ids}
\end{Highlighting}
\end{Shaded}

\begin{verbatim}
[11, 10, 9, 8, 7, 6, 5, 4, 3, 2, 1, 0, -1]
\end{verbatim}

Any value in the list can also be removed via \texttt{remove()}. Note
that, \texttt{remove()} only removes the first occurence of a value.

\begin{Shaded}
\begin{Highlighting}[]
\NormalTok{ids.remove(}\DecValTok{8}\NormalTok{)}
\NormalTok{ids}
\end{Highlighting}
\end{Shaded}

\begin{verbatim}
[11, 10, 9, 7, 6, 5, 4, 3, 2, 1, 0, -1]
\end{verbatim}

You can get the methods you can call upon a \textbf{list} using the
\texttt{dir()} function

\begin{Shaded}
\begin{Highlighting}[]
\BuiltInTok{dir}\NormalTok{(}\BuiltInTok{list}\NormalTok{)[}\OperatorTok{{-}}\DecValTok{10}\NormalTok{:] }\CommentTok{\# [{-}10:] gives the last 10 elements of the list returned by dir()}
\end{Highlighting}
\end{Shaded}

\begin{verbatim}
['clear',
 'copy',
 'count',
 'extend',
 'index',
 'insert',
 'pop',
 'remove',
 'reverse',
 'sort']
\end{verbatim}

\section{Built-in Functions to use with
lists}\label{built-in-functions-to-use-with-lists}

Python has built-in \texttt{min()} and \texttt{max()} functions.

\begin{Shaded}
\begin{Highlighting}[]
\BuiltInTok{max}\NormalTok{(ids)}
\end{Highlighting}
\end{Shaded}

\begin{verbatim}
11
\end{verbatim}

\begin{Shaded}
\begin{Highlighting}[]
\BuiltInTok{min}\NormalTok{(ids)}
\end{Highlighting}
\end{Shaded}

\begin{verbatim}
-1
\end{verbatim}

Summation of all elements in the list are also possible.

\begin{Shaded}
\begin{Highlighting}[]
\BuiltInTok{sum}\NormalTok{(ids)}
\end{Highlighting}
\end{Shaded}

\begin{verbatim}
57
\end{verbatim}

and the number of elements (length) in a list can be accessed via
\texttt{len()} function.

\begin{Shaded}
\begin{Highlighting}[]
\BuiltInTok{len}\NormalTok{(ids)}
\end{Highlighting}
\end{Shaded}

\begin{verbatim}
12
\end{verbatim}

Note that, \texttt{min()}, \texttt{max()}, and \texttt{len()} are not
\texttt{methods} for \texttt{list} \textbf{Class} and they can be used
with other collections of items such as \textbf{tuples} and
\textbf{arrays}.

Lists objects have also be sorted via \texttt{sort()} method.

\begin{Shaded}
\begin{Highlighting}[]
\NormalTok{random\_numbers }\OperatorTok{=}\NormalTok{ [}\DecValTok{5353}\NormalTok{, }\DecValTok{314235}\NormalTok{, }\DecValTok{353}\NormalTok{, }\DecValTok{9}\NormalTok{,}\OperatorTok{{-}}\DecValTok{12}\NormalTok{]}
\NormalTok{random\_numbers.sort()}
\NormalTok{random\_numbers}
\end{Highlighting}
\end{Shaded}

\begin{verbatim}
[-12, 9, 353, 5353, 314235]
\end{verbatim}

Lists don't have to be collection of the same type values. \textbf{Any}
Python object can be stored in a list including other lists.

\begin{Shaded}
\begin{Highlighting}[]
\NormalTok{mixed }\OperatorTok{=}\NormalTok{ [}\StringTok{"a"}\NormalTok{, }\StringTok{"b"}\NormalTok{, }\StringTok{"c"}\NormalTok{, }\DecValTok{1}\NormalTok{, }\DecValTok{10}\NormalTok{ , }\DecValTok{100}\NormalTok{, }\VariableTok{True}\NormalTok{, }\VariableTok{False}\NormalTok{, ids, }\VariableTok{None}\NormalTok{, }\FloatTok{3.14}\NormalTok{, random\_numbers, }\StringTok{"d"}\NormalTok{, }\StringTok{"e"}\NormalTok{, }\StringTok{"f"}\NormalTok{]}
\NormalTok{mixed}
\end{Highlighting}
\end{Shaded}

\begin{verbatim}
['a',
 'b',
 'c',
 1,
 10,
 100,
 True,
 False,
 [11, 10, 9, 7, 6, 5, 4, 3, 2, 1, 0, -1],
 None,
 3.14,
 [-12, 9, 353, 5353, 314235],
 'd',
 'e',
 'f']
\end{verbatim}

Obiviously, you can not call \texttt{sort()}, \texttt{min()},
\texttt{max()} or similar methods on a mixed list.

\section{Indexing}\label{indexing}

Like most programming languages index start from \texttt{0} in Python.
And the index should put within \texttt{{[}{]}} next to the list name.

\begin{Shaded}
\begin{Highlighting}[]
\NormalTok{mixed[}\DecValTok{0}\NormalTok{] }\CommentTok{\# this will return the first element of a list }
\end{Highlighting}
\end{Shaded}

\begin{verbatim}
'a'
\end{verbatim}

To access the last element in a list. You can use \texttt{-1} as the
index. Or \texttt{-2}, \texttt{-3} so on as a way of starting from the
end.

\begin{Shaded}
\begin{Highlighting}[]
\NormalTok{mixed[}\OperatorTok{{-}}\DecValTok{1}\NormalTok{]}
\end{Highlighting}
\end{Shaded}

\begin{verbatim}
'f'
\end{verbatim}

is equivalent to

\begin{Shaded}
\begin{Highlighting}[]
\NormalTok{mixed[}\BuiltInTok{len}\NormalTok{(mixed) }\OperatorTok{{-}} \DecValTok{1}\NormalTok{]}
\end{Highlighting}
\end{Shaded}

\begin{verbatim}
'f'
\end{verbatim}

\section{Slicing}\label{slicing}

Let's say you want to get the first element on a list. You can use the
\texttt{:5} (equivalent of \texttt{0:5}) for indexing that will return
the first five elements.

\begin{Shaded}
\begin{Highlighting}[]
\NormalTok{mixed[:}\DecValTok{5}\NormalTok{]}
\end{Highlighting}
\end{Shaded}

\begin{verbatim}
['a', 'b', 'c', 1, 10]
\end{verbatim}

Or you can get next 3 elements

\begin{Shaded}
\begin{Highlighting}[]
\NormalTok{mixed[}\DecValTok{5}\NormalTok{:}\DecValTok{8}\NormalTok{] }\CommentTok{\# will return the elements from index 5 to index 8, where 8 is not included.}
\end{Highlighting}
\end{Shaded}

\begin{verbatim}
[100, True, False]
\end{verbatim}

Or the last 5 elements

\begin{Shaded}
\begin{Highlighting}[]
\NormalTok{mixed[}\OperatorTok{{-}}\DecValTok{5}\NormalTok{:]}
\end{Highlighting}
\end{Shaded}

\begin{verbatim}
[3.14, [-12, 9, 353, 5353, 314235], 'd', 'e', 'f']
\end{verbatim}

Or you might want the every second element. You can use the \texttt{::2}
which means from \textbf{beginning to end} (\texttt{::}) every 2nd
(\texttt{2}).

\begin{Shaded}
\begin{Highlighting}[]
\NormalTok{mixed[::}\DecValTok{2}\NormalTok{]}
\end{Highlighting}
\end{Shaded}

\begin{verbatim}
['a', 'c', 10, True, [11, 10, 9, 7, 6, 5, 4, 3, 2, 1, 0, -1], 3.14, 'd', 'f']
\end{verbatim}

Of course 2 can be changed with any number of your choosing.

\begin{Shaded}
\begin{Highlighting}[]
\NormalTok{mixed[::}\DecValTok{5}\NormalTok{]}
\end{Highlighting}
\end{Shaded}

\begin{verbatim}
['a', 100, 3.14]
\end{verbatim}

Or you can get them in reverse order, basically reversing the list
\texttt{::-1}

\begin{Shaded}
\begin{Highlighting}[]
\NormalTok{mixed[::}\OperatorTok{{-}}\DecValTok{1}\NormalTok{]}
\end{Highlighting}
\end{Shaded}

\begin{verbatim}
['f',
 'e',
 'd',
 [-12, 9, 353, 5353, 314235],
 3.14,
 None,
 [11, 10, 9, 7, 6, 5, 4, 3, 2, 1, 0, -1],
 False,
 True,
 100,
 10,
 1,
 'c',
 'b',
 'a']
\end{verbatim}

\chapter{Tuples}\label{tuples}

\section{todo}\label{todo}

\chapter{Dictionaries}\label{dictionaries}

\section{TODO}\label{todo-1}




\end{document}
